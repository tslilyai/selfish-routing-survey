\subsection{Altruism and Spite}
We first consider an alternative model proposed by Chen and Kempe in 2008~\cite{chen}, which assumes that users are ``not entirely selfish.''
Chen and Kempe note that social experiments from both economic and psychology have shown humans do not behave rationally in a selfish manner; instead, our behavior is better modeled as either altruistic or malicious (spiteful).
Their model proposes a simple way to capture how people make decisions based upon how much latency a particular decision will cost other users: if someone is spiteful, she will want to increase others' latencies, and if she is altruistic, she will want to decrease their latencies.

\subsubsection{Formalization}
The formal Chen and Kempe model introduces a per-user \emph{altruism} coefficient $\beta$ and an individual user cost function
of $c^\beta_p$ for all paths $p$:
$$c^\beta_p(f) = \sum_{e \in p} \ell_e(f_e) + \beta\sum_{e\in P} f_e\ell'_e(f_e)$$
where $\ell_e(\cdot)$ is the latency function from the selfish routing setting, and $\ell'_e(\cdot)$ is the derivative with respect to $f_e$.

Note that the first term is exactly the individual user cost function from the selfish routing model (and thus the two models are equivalent when $\beta = 0$). The second term corresponds to the derivative of the social welfare latency cost on $p$ and is weighed by $\beta$; we use the derivative, rather than the value, of the social welfare cost on $p$ because each user only controls an infinitesimally small amount of the flow: if we were to use the value, a single user's choice would have no effect on the social welfare cost! 
Instead, a user can account for how she will affect the social welfare cost via the rate at which her choice of path affects other users.

If $\beta$ is negative, a user is spiteful: we know that adding a little more flow to $p$ will increase the social welfare cost of taking $p$ (the derivative $\ell'_e$ is positive), and since we negate this value, this lowers the user's perceived cost of taking $p$.
Conversely, if $\beta$ is positive, a user is altruistic: increasing flow increases the social welfare cost on $p$ and also the user's perceived cost of taking $p$.
We assume that $\beta$ ranges from -1 (extremely spiteful) to 1 (extremely altruistic), where $\beta=0$ corresponds to selfishness.
All analysis of the model assumes a particular distribution $\psi$ of $\beta$ for all users. 

Note that we can compare this model to the selfish model using the price of anarchy as a measure of inefficiency
because the altruistic model still achieves Nash equilibrium for any $\psi$ and cost function $c^\beta_p$. (Given any $\psi$, $c^\beta_p$ are continuous in the choice of path $p$ and in the distribution of other users' strategies $f$: Mas-Colell~\cite{mascolell} showed that any game of infinitely many players with cost functions continuous in the actions of the players and distribution of actions by other players has a Nash equilibrium.)
%Nash equilibrium is achieved at the flow solutions to the program
%$$NE^\beta = \min_f\sum_e\int_0^{f_e}c_e^\beta(t)dt \text{ subject to feasibility constraints}$$
We next present Chen and Kempe's core results about the price of anarchy of arbitrary networks when $\psi$ is uniform (all users have the same $\beta$ value), and briefly mention their results of non-uniform $\psi$ in parallel-link networks.

\subsubsection{Uniformly Distributed Altruism}
We first consider the case where $\psi$ is uniformly distributed, such that $\beta$ and therefore $c^\beta_p$ is the same for each user. We additionally assume that users tend to be altruistic, i.e., $\beta > 0$.
With these assumptions, we get a nice bound on the price of anarchy:
\begin{theorem}
For any $G$, demand rates $r$, and 
a uniform distribution $\psi$ with $\beta \in (0, 1]$,
if $\ell_e$ is nondecreasing and convex for all $e$, then the price of anarchy is bounded by 
    $$\rho(G,r,\ell,\psi) \le \frac{1}{\beta}$$
\end{theorem}

\begin{proof-sketch}
    Consider the two (convex) functions that we minimize for each of the user objectives
    for the altruistic model and the social welfare model (a cooperative model optimizing for the total overall latency).
    For simplicity, let $B(f)$ be the function minimized in the altruistic model; the second function is simply our social welfare cost $C(f)$.
    We can write these and manipulate them into comparable forms as follows:
    $$B(f) = \sum_e\int_0^{{f}_e}\Big(c_e^\beta(t)\Big)dt = 
        \sum_e\int_0^{{f}_e} \Big(\ell_e(t) + \beta t\ell'e(t)\Big)dt\text{ (by definition of $c^\beta_e$)}$$
    $$C(f) = \sum_ef_e\ell_e(f_e) = \sum_e\int_0^{f_e} \Big((t\ell_e(t))' \Big)dt 
        = \sum_e\int_0^{f_e} \Big(\ell_e(t) + tl'e(t)\Big)dt$$ 
    It is clear that for any feasible flow $f$, 
    $B(f) \le C(f) \le \frac{B(f)}{\beta} \text{ because $\beta\in(0,1]$}$.
    We now let $\hat{f}$ be the flow at Nash equilibrium and $f^*$ be the flow at optimum social welfare. Because these are the optimal flows for their respective objectives, we know that 
    $C(\hat{f}) \le \frac{B(\hat{f})}{\beta} \le \frac{B(f^*)}{\beta} \le \frac{C(f^*)}{\beta}$,
    proving that 
    $\rho(G,r,\ell,\psi) \le \frac{1}{\beta}$
\end{proof-sketch}

\subsubsection{Uniformly Distributed Spite}
Chen and Kempe then address the problem of spite: how (uniformly) spiteful can users be before $\rho$ becomes infinite?
It turns out that this depends on the type of latency function! Our analysis begins by reasoning about the price of anarchy of a given class ${L}$ of latency functions, 
$\rho(G,r,{L},\psi)$.

We find that the price of anarchy of a class of functions $L$ is lower-bounded by the worst possible price of anarchy (over all $\ell \in L$ ) achieved in a two-link, two-node network (such as in Figure~\ref{fig:Pigou}) routing $r$ demand with the following latency functions on its two edges:
$\ell_1(x) = \ell(x)$ and $\ell_2(x) = c^\beta{r} = \ell(r) +\beta r\ell'(r)$.

We refer to this specific two-link problem for a given input $(G,r,L,\psi)$ as $T_\beta$, and the maximum price of anarchy of this problem over all $\ell \in L$ as $\rho(T_\beta)$.
\begin{theorem}
For any $G$, demand rates $r$ and uniform distribution $\psi$ of $\beta \in (-1, 1]$,
    $$\rho(G,r,{L},\psi) \le \rho(T_\beta)$$
   i.e., the price of anarchy of a class of functions $L$ routing $r$ flow in $G$ is 
    bounded by the price of anarchy routing $r$ flow through the network $T_\beta$.
    \end{theorem}
\begin{proof-sketch}
    We give a brief overview of the proof technique here.
    Note that by definition, 
    the flow in $T_\beta$ at Nash equilibrium will route all demand $r$ on the edge with latency function $\ell_1(x) = c(x)$, whereas the socially optimal flow will route some flow $x$ on the edge with latency function $\ell_2 = c^\beta(r)$. This gives us the worst case for $\rho(T_\beta)$ given any $\ell \in L$ as
    $$\rho(T_\beta) = \max_{l\in{L}} \max_{x,r\ge 0} \frac{r\ell(r)}{x\ell(x) + (r-x)(c^\beta(r))}$$
    The proof proceeds by considering social welfare cost of the flow $f^*$ optimizing $C$. By unfolding the definition of $\rho(T_\beta$) to get a bound for $x\ell_e(x)$ for arbitrary $x$ and $r$, we can then apply this bound to each edge with $x = f^*_e$ and $r = \hat{f}_e$, where $\hat{f}$ is the optimizing flow at Nash equilibrium. 
    $$\forall x,r\ge 0,~ x\ell_e(x) \ge \frac{r\ell_e(r)}{\rho(T_\beta)}  + (x-r)\ell^\beta_e(r)
    \implies  f^*_e\ell_e(f^*_e) \ge \frac{\hat{f}_e\ell_e(\hat{f}_e)}{\rho(T_\beta)}  + (f^*_e-\hat{f}_e)\ell^\beta_e(\hat{f}_e)$$
    With some mathematical manipulation, we can derive a comparison of $C(f^*)$ to $B(\hat{f})$ satisfying the above bound:
    \begin{align*}
        C(f^*) = \sum_e f^*_e\ell_e(f^*_e) &\ge \sum_e \frac{\hat{f}_e\ell_e(\hat{f}_e)}{\rho(T_\beta)} + (f^*_e-\hat{f}_e)\ell^\beta_e(\hat{f}_e)\\
        &\ge \frac{C(\hat{f})}{\rho(T_\beta)} + \sum_e (f^*_e-\hat{f}_e)\ell^\beta_e(\hat{f}_e)
    \end{align*}
    Note that $\sum_e (f^*_e) \ell^\beta_e(\hat{f}_e) \ge \sum_e \hat{f}_e\ell^\beta_e(\hat{f}_e)$ by definition of $\hat{f}$ being a flow at Nash equilibrium (see Theorem~\ref{variational}).
    Thus, we get that 
        $$C(f^*) \ge \frac{C(\hat{f})}{\rho(T_\beta)}\implies \rho(T_\beta) \ge \frac{C(\hat{f})}{C(f^*)}$$
\end{proof-sketch}

Since we know how to bound $\rho(G,r,{L},\psi)$ by the (uniform) value of $\beta$, we can now determine at which values of $\beta$ this lower bound is infinite: how spiteful do users have to be to cause each other infinitely more suffering? The following result shows that if the latency functions are in the class $L_d =$ polynomials of degree $\le d$, the price of anarchy is bounded when $\beta$ is at least $\frac{-1}{d}$ (and is infinite when $\beta < \frac{-1}d$).
\begin{theorem}
    Let $L_d$ be the class of latency functions that are polynomials of degree $\le d$.
    For any $G$, demand rates $r$, $\ell \in L_d$, and uniform distribution $\psi$ with $\beta \in (\frac{-1}{d}, 1]$,
    $$\rho(G,r,\ell,\psi) \le \Big(\Big(\frac{1+\beta d}{1+d}\Big)^{1/d}\Big(\frac{1+\beta d}{1+d} + 1 + \beta d\Big)+ 1 + \beta d\Big)^{-1}$$
\end{theorem}
\begin{proof-sketch}
From Theorem 4, we know that 
$\rho(G,r,{L},\psi)$% \le \max_{l\in{L_d}} \max_{x,r\ge 0} \frac{r\ell(r)}{x\ell(x) + (r-x)(c^\beta(r))}$, where the RHS corresponds to 
    is bounded above by $\rho(T_\beta)$. Thus, we only need to consider how bad $\rho(T_\beta)$ can get given any $l\in L_d$.

    The key observation is that $\exists\lambda \in [0,1]$ s.t. $c^1(r\lambda) = c^\beta(r)$, where $c^1$ is the cost function with uniform altruism value $\beta=1$. This will allow us to reason about the amount of flow $r\lambda$ that the optimum solution will place on the link with latency function $l_1(x) = \ell(x)$, and precisely compute the price of anarchy in terms of $\beta$.

    Observe that a Nash equilibrium flow in $T_\beta$ routing $r$ units from the source to the destination will put all flow on the link with cost function $l_1(x) = \ell(x)$. By definition of $\lambda$, the solution optimizing social welfare will put $r\lambda$ flow on the first link with latency function $\ell_1(x) = \ell(x)$, and $r(1-\lambda)$ on the second link with latency function $\ell_2(x) = c^\beta(r)$. This gives us a a bound on the price of anarchy:
    $$\rho(G,r\lambda,\ell,\psi) \le \rho(T_\beta) \le \Big(\frac{\lambda \ell(r\lambda)}{\ell(r)} + \Big(1-\lambda\Big)\Big(1+\frac{\beta r\ell'(r)}{\ell(r)}\Big)\Big)^{-1}$$

    To satisfy our observation, we need to make an appropriate choice of $\lambda$. To simplify our calculation of $\lambda$, we can, without loss of generality, consider latency functions $\ell(x) = ax^i$ for some $i \le d$ to represent all functions $\ell \in L_d$. We can then solve $c^1(r\lambda) = c^\beta(r)$ for $\lambda$ to get $\lambda = \Big(\frac{1+\beta i}{1+i}\Big)^{\frac{1}{i}}$.
    With this $\lambda$, we get that $\frac{\ell(r\lambda)}{\ell(r)} = \frac{1+\beta i}{1+i}$ and $\frac{\ell'(r)}{\ell(r)} = \frac{i}{r}$. We can plug these values into the bound to get:
    $$\rho(G,r,\ell,\psi) = \rho(G,\lambda r, \ell,\beta=1) \le \Big(\Big(\frac{1+\beta i}{1+i}\Big)^{1/i}\Big(\frac{1+\beta i}{1+i} + 1 + \beta i\Big)+ 1 + \beta i\Big)^{-1}$$
    This is increasing in $i$, giving us the worst-case bound when $i=d$:
$$\Big(\Big(\frac{1+\beta d}{1+d}\Big)^{1/d}\Big(\frac{1+\beta d}{1+d} + 1 + \beta d\Big)+ 1 + \beta d\Big)^{-1}$$
\end{proof-sketch}

\subsubsection{Arbitrarily Distributed Altruism}
Chen and Kempe go on to extend their analysis to when users have an arbitrary distribution $\psi$ of altruism (with no spiteful users) in \emph{parallel link networks}, networks with only two nodes (a source and destination) and parallel edges running between the nodes. 
We briefly mention their results here, but direct the reader to the paper for a more detailed proof.
%, which have only one demand source-sink pair $(s,t)$ and (parallel) edges only between $s$ and $t$.
%Stackleberg routing shown to have unlimited PoA in single-commodity networks
Their main result mirrors that of uniform altruism:
\begin{theorem}
    Given any parallel link network $G$, demand rates $r$, altruism density function $\psi$ with average altruism $\bar{\beta}$ and non-negative support (no spiteful users), and convex and non-decreasing cost functions $\ell_e$,
   $\rho(G,r,\ell,\psi) \le \frac{1}{\bar{\beta}}$
\end{theorem}
Finally, Chen and Kempe comment on the restriction to $\beta \ge 0$: if even a small fraction of users are spiteful, these users can, in some networks instances, lead to an price of anarchy much greater than 1 regardless of the value of $\bar{\beta}$.
