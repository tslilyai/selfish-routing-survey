\subsection{Altruistic}
The first alternative model we consider is that proposed by Chen and Kempe in 2008~\cite{chen}, which assumes that users are ``not entirely selfish.''
Chen and Kempe note that social experiments from both economic and psychology have shown humans do not behave rationally in a selfish manner; instead, our behavior is better modeled as either altruistic or malicious (spiteful).
Their model proposes a simple way to capture how people make decisions based upon how much cost (latency) a particular decision will cost other users; if someone is spiteful, she will want to increase their cost, and if she are altruistic, she will want to decrease their cost.

\paragraph{Formalization}
More formally, the Chen and Kempe model introduces an \emph{altruism} coefficient $\beta$ and a new cost function
$c^\beta_p$ for all paths $p$: 
$$c^\beta_p(f) = \sum_{e \in p} c_e(f_e) + \beta\sum_{e\in P} f_ec'_e(f_e)$$
where $c_e(\cdot)$ is the cost function from the selfish routing setting, and $c'_e(\cdot)$ is the derivative with respect to $f_e$.

Note that the first term is exactly the cost used in the selfish routing model (and thus is equivalent to an altruism coefficient of $\beta = 0$).
The second term corresponds to the derivative of the social welfare cost on $p$ and is weighted by $\beta$; intuitively, since we are dealing with an infinite number of users each controlling an infinitisimally small amount of the flow, we model the effect one user has on another 
via the rate (derivative) at which that user's choice of path affects other users.

If $\beta$ is negative, a user is spiteful: we know that adding a little more flow to $p$ will increase the social welfare cost of taking $p$ (the derivative $c'_e$ is positive), and since we negate this value, this lowers the user's perceived cost of taking $p$.
Conversely, if $\beta$ is positive, a user is altruistic: increasing flow increases the social welfare cost on $p$ and also the user's perceived cost of taking $p$.

\subsubsection{Results}
