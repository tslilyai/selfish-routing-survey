\subsection{Altruism and Spite}
The first alternative model we consider is that proposed by Chen and Kempe in 2008~\cite{chen}, which assumes that users are ``not entirely selfish.''
Chen and Kempe note that social experiments from both economic and psychology have shown humans do not behave rationally in a selfish manner; instead, our behavior is better modeled as either altruistic or malicious (spiteful).
Their model proposes a simple way to capture how people make decisions based upon how much latency a particular decision will cost other users; if someone is spiteful, she will want to increase others' latencies, and if she are altruistic, she will want to decrease their latencies.

\subsubsection{Formalization}
The formal Chen and Kempe model introduces a per-user \emph{altruism} coefficient $\beta$ and a new individual user cost function
$c^\beta_p$ for all paths $p$: 
$$c^\beta_p(f) = \sum_{e \in p} l_e(f_e) + \beta\sum_{e\in P} f_el'_e(f_e)$$
where $l_e(\cdot)$ is the latency function from the selfish routing setting, and $l'_e(\cdot)$ is the derivative with respect to $f_e$.

Note that the first term is exactly the user cost used in the selfish routing model (and thus the two user costs are equivalent when $\beta = 0$).
The second term corresponds to the derivative of the social welfare latency cost on $p$ and is weighed by $\beta$; we use the derivative, rather than the value, of the social welfare cost on $p$ because each user only controls an infinitesimally small amount of the flow. Thus, if we were to use the value, a single user's choice would have no effect on the social welfare cost! 
Instead, a user can account for how she will affect the social welfare cost via the rate at which her  choice of path affects other users.

If $\beta$ is negative, a user is spiteful: we know that adding a little more flow to $p$ will increase the social welfare cost of taking $p$ (the derivative $l'_e$ is positive), and since we negate this value, this lowers the user's perceived cost of taking $p$.
Conversely, if $\beta$ is positive, a user is altruistic: increasing flow increases the social welfare cost on $p$ and also the user's perceived cost of taking $p$.
We assume that $\beta$ ranges from -1 (extremely spiteful) to 1 (extremely altruistic), where $\beta=0$ corresponds to selfishness.

All analysis of the model assumes a particular distribution $\psi$ of $\beta$ for all users. 
Futhermore, we can compare this model to the selfish model using the PoA as a measure of inefficiency
because the atruistic model still achieves Nash equilibrium for any $\psi$ and cost function $c^\beta_p$: given any $\psi$, $c^\beta_p$ are continuous in the choice of path $p$, and we can apply the result of Theorem 1 of Mas-Colell~\cite{mascolell}. Nash equilibrium is achieved at the flow solutions to the program
$$NE_\beta = \min_f\sum_e\int_0^{f_e}c_e^\beta(t)dt \text{ subject to feasibility constraints}$$
We next present Chen and Kempe's core results about the PoA of arbitrary networks when $\psi$ is uniform (all users have the same $\beta$ value), and briefly mention their results of non-uniform $\psi$ in parallel-link networks.

\subsubsection{Uniformly Distributed Altruism}
We first consider the case where $\psi$ is uniformly distributed, such that $\beta$ and therefore $c^\beta_p$ is the same for each user. We additionally assume that users tend to be altruistic, i.e., $\beta > 0$.
\begin{theorem}
For any $G$, demand rates $r$, and 
a uniform distribution $\psi$ with $\beta \in (0, 1]$,
if $l_e$ is nondecreasing and convex for all $e$, then the price of anarchy is bounded by 
    $$\rho(G,r,l,\psi) \le \frac{1}{\beta}$$
\end{theorem}

\begin{proof-sketch}
    Consider the two (convex) functions that we minimize for each of the two objectives $NE_\beta$ and $SW$. For simplicity, let $B(f)$ be the function minimized in $NE_\beta$; the second is our social welfare cost $C(f)$.
    We can write these and manipulate them into comparable forms as follows:
    $$B(f) = \sum_e\int_0^{{f}_e}c_e^\beta(t)dt = 
        \sum_e\int_0^{{f}_e} l_e(t) + \beta tl'e(t)dt\text{ (by definition of $c^\beta_e$)}$$
    $$C(f) = \sum_ef_el_e(f_e) = \sum_e\int_0^{f_e} (tl_e(t))' dt 
        = \sum_e\int_0^{f_e} l_e(t) + tl'e(t)dt$$ 
    It is clear that for any feasible flow $f$, 
    $B(f) \le C(f) \le \frac{B(f)}{\beta} \text{ because $\beta\in(0,1]$}$.
    We now let $\hat{f}$ be the flow at Nash Equilibrium and $f^*$ be the flow at optimum social welfare. Because these are the optimal flows for their respective objectives, we know that 
    $C(\hat{f}) \le \frac{B(\hat{f})}{\beta} \le \frac{B(f^*)}{\beta} \le \frac{C(f^*)}{\beta}$,
    proving that 
    $\rho(G,r,l,\psi) \le \frac{1}{\beta}$
\end{proof-sketch}

\subsubsection{Uniformly Distributed Spite}
Chen and Kempe then address the problem of spite: how (uniformly) spiteful can users be before the PoA becomes infinite?
It turns out that this depends on the type of latency function! Our analysis begins by reasoning about the PoA of a given class ${L}$ of latency functions: 
$$\rho(G,r,{L},\psi)$$ 

It turns out that the PoA of a class of functions $L$ is lower-bounded by the worst PoA achieved in a two-link, two-node network (such as in Figure~\ref{fig:Pigou}) with demand rate $r$ and latency function $l_1(x) = l(x)$ and constant latency function $l_2(x) = c^\beta{r}$, for any choice of $l \in L$. We refer to this specific problem as $T_\beta$ given the user cost function $c^\beta$, and the price of anarchy of this problem as $\rho(T_\beta)$.
\begin{theorem}
For any $G$, demand rates $r$ and uniform distribution $\psi$ of $\beta \in (-1, 1]$,
    $$\rho(G,r,{L},\psi) \le \rho(T_\beta)$$
    In other words, the price of anarchy of a class of functions $L$ routing $r$ flow in $G$ is 
    bounded by the price of anarchy routing $r$ flow through the network $T_\beta$.
    \end{theorem}
\begin{proof-sketch}
    We give a brief overview of the proof technique here.
    Note that $$\rho(T_\beta) = \max_{l\in{L}} \max_{x,r\ge 0} \frac{rl(r)}{xl(x) + (r-x)(c^\beta(r))}$$
    The proof proceeds by considering social welfare cost of the flow $f^*$ optimizing $C$. By unfolding the RHS part of the above goal to get a bound for $xl_e(x)$, we can then apply this bound to each edge with $x = f^*_e$ and $r = \hat{f}_e$, where $\hat{f}$ is the optimizing flow at Nash Equilibrium. With some mathematical manipulation, we can derive a comparison of $C(f^*)$ to $B(\hat{f})$ satisfying the above bound.
\end{proof-sketch}

Since we know how to bound $\rho(G,r,{L},\psi)$ by the (uniform) value of $\beta$, we can now determine at which values of $\beta$ this lower bound is infinite: how spiteful do users have to be to cause each other infinitely more suffering? The following result shows that if the cost functions are in $L_d =$ polynomials of degree $\le d$, the PoA is bounded when $\beta$ is at least $\frac{-1}{d}$ (and is infinite when $\beta < \frac{-1}d$).
\begin{theorem}
For any $G$, demand rates $r$, $l_e \in L_d$,
and uniform distribution $\psi$ with $\beta \in (\frac{-1}{d}, 1]$,
    $$\rho(G,r,l,\psi) \le \Big(\Big(\frac{1+\beta d}{1+d}\Big)^{1/d}\Big(\frac{1+\beta d}{1+d} + 1 + \beta d\Big)+ 1 + \beta d\Big)^{-1}$$
\end{theorem}
\begin{proof-sketch}
From Theorem 4, we know that 
$\rho(G,r,{L},\psi)$% \le \max_{l\in{L_d}} \max_{x,r\ge 0} \frac{rl(r)}{xl(x) + (r-x)(c^\beta(r))}$, where the RHS corresponds to 
 is bounded above by the worst-case PoA for any cost function $l \in L_d$ in $T_\beta$, a two-node, two-link network with cost functions $l_1(x) = l(x)$ and $l_2(x) = c^\beta(r)$.
Thus, we only need to consider how bad $\rho(T_\beta)$ can get given any $l\in L_d$.

The key observation is that $\exists\lambda \in [0,1]$ s.t. $c^1(r\lambda) = c^\beta(r)$, where $c^1$ is the cost function with uniform altruism value $\beta=1$. Our strategy is three-part: 
\begin{enumerate}
    \item We bound the price of anarchy of routing $r\lambda$ units of flow at Nash Equilibrium with user cost function $c^\beta$ in $T_\beta$
   \item We use the bound from step 1 to get an upper bound on the price of anarchy routing $r\lambda$ units of flow in $T_\beta$ when the user cost function is $c^1$. since by Theorem 3, increasing $\beta$ to a maximum of 1 will only decrease the price of anarchy. 
    \item Drawing on our key observation, the bound for $c^1$ routing $r\lambda$ units of flow can be applied to bound the price of anarchy routing $r$ units of flow when user costs are $c^\beta$.
We only need to make an appropriate choice of $\lambda$ to make our observation valid.
\end{enumerate}

We start with step (1): A Nash Equilibrium flow routing $r\lambda$ units from the source to the destination will put $r\lambda$ flow on the first link with $l_1 = l(x)$, and the rest on the second link with $l_2 = c^\beta(x)$. The solution optimizing social welfare will put all flow on the first link with latency function $l_1 = l(x)$.
This gives us a a bound on the price of anarchy when $r\lambda$ units of flow are sent through $T_\beta$ 
    $$\rho(G,r\lambda,l,\psi) \le \rho(T_\beta) \le \Big(\frac{\lambda l(r\lambda)}{l(r)} + \Big(1-\lambda\Big)\Big(1+\frac{\beta rl'(r)}{l(r)}\Big)\Big)^{-1}$$
    
    This allows us to perform step (2): using Theorem 3, the above bound from step (1) also bounds the price of anarchy when the user cost function is $c^1$, i.e. $\psi$ is a uniform distribution when $\beta = 1$
    $$\rho(G,\lambda r,l,\beta=1) \le \rho(G,r\lambda, l, \psi) \le \Big(\frac{\lambda l(r\lambda)}{l(r)} + \Big(1-\lambda\Big)\Big(1+\frac{\beta rl'(r)}{l(r)}\Big)\Big)^{-1}$$

    We now can perform step (3). First, we need to make an appropriate choice of $\lambda$. Without loss of generality we can consider latency functions $l(x) = ax^i$ for some $i \le d$ to represent all functions $l \in L_d$. For these latency functions $l(x) = ax^i$, we know that 
    $c^1(r\lambda) = c^\beta(r)$ when $\lambda = \Big(\frac{1+\beta i}{1+i}\Big)^{\frac{1}{i}}$.
    With this $\lambda$, we get that $\frac{l(r\lambda)}{l(r)} = \frac{1+\beta i}{1+i}$ and $\frac{l'(r)}{l(r)} = \frac{i}{r}$. We can plug these values into the bound to get:
    $$\rho(G,r,l,\psi) = \rho(G,\lambda r, l,\beta=1) \le \Big(\Big(\frac{1+\beta i}{1+i}\Big)^{1/i}\Big(\frac{1+\beta i}{1+i} + 1 + \beta i\Big)+ 1 + \beta i\Big)^{-1}$$
    This is increasing in $i$, giving us the worst-case bound when $i=d$:
$$\Big(\Big(\frac{1+\beta d}{1+d}\Big)^{1/d}\Big(\frac{1+\beta d}{1+d} + 1 + \beta d\Big)+ 1 + \beta d\Big)^{-1}$$
\end{proof-sketch}

\subsubsection{Arbitrarily Distributed Altruism}
Chen and Kempe go on to extend their analysis to when users have an arbitrary distribution $\psi$ of altruism (with no spiteful users) in \emph{parallel link networks}. We briefly mention their results here, but direct the reader to the paper for a more detailed proof and explanation. 
%, which have only one demand source-sink pair $(s,t)$ and (parallel) edges only between $s$ and $t$.
%Stackleberg routing shown to have unlimited PoA in single-commodity networks
Their main result mirrors that of the PoA with uniform altruism:
\begin{theorem}
Given any parallel link network $G$, demand rates $r$, altruism density function $\psi$ with average altruism $\bar{\beta}$ and non-negative support, and convex and non-decreasing cost functions $l_e$,
   $\rho(G,r,l,\psi) \le \frac{1}{\bar{\beta}}$
\end{theorem}
One interesting observation is that this theorem leads to the following corrollary: because the theorem applies for a distribution in which a rate of $\bar{\beta}$ of users are completely altruistic and 1-$\bar{\beta}$ users are completely selfish, this is equivalent to $\bar{\beta}$ of the population being under centralized (coordinated) control:
\begin{corollary}
Given any parallel link network $G$, demand rates $r$, altruism density function $\psi$ with average altruism $\bar{\beta}$ and non-negative support, and convex and non-decreasing cost functions $l_e$,
if $\bar{\beta}$ fraction of traffic is controlled by a central authority, then
   $\rho(G,r,l,\psi) \le \frac{1}{\bar{\beta}}$
\end{corollary}
This is exactly the result proven by Roughgarden in 2004~\cite{roughgarden04}, and provides an interesting connection between models that include altruism and models of centralized/coordinated traffic control.
%Almost all of the latency is incurred by a small fraction of spiteful users who together
%congest a link with very steep increase. At the same time, all altruistic users use links with very small constant latency.
%Then, the PoA is much larger than 1, while the bounds of both theorems would require it to be close to 1
