\subsection{Altruism and Spite}
The first alternative model we consider is that proposed by Chen and Kempe in 2008~\cite{chen}, which assumes that users are ``not entirely selfish.''
Chen and Kempe note that social experiments from both economic and psychology have shown humans do not behave rationally in a selfish manner; instead, our behavior is better modeled as either altruistic or malicious (spiteful).
Their model proposes a simple way to capture how people make decisions based upon how much cost (latency) a particular decision will cost other users; if someone is spiteful, she will want to increase their cost, and if she are altruistic, she will want to decrease their cost.

\subsubsection{Formalization}
The formal Chen and Kempe model introduces a per-user \emph{altruism} coefficient $\beta$ and a new cost function
$c^\beta_p$ for all paths $p$: 
$$c^\beta_p(f) = \sum_{e \in p} c_e(f_e) + \beta\sum_{e\in P} f_ec'_e(f_e)$$
where $c_e(\cdot)$ is the cost function from the selfish routing setting, and $c'_e(\cdot)$ is the derivative with respect to $f_e$.

Note that the first term is exactly the cost used in the selfish routing model (and thus is equivalent to an altruism coefficient of $\beta = 0$).
The second term corresponds to the derivative of the social welfare cost on $p$ and is weighed by $\beta$; we use the derivative, rather than the value, of the social welfare cost on $p$ because each user only controls an infinitesimally small amount of the flow. Thus, if we were to use the value, 
a single user's choice would have no effect on the social welfare cost! 
Instead, a user can account for how she will affect the social welfare cost via the rate at which her  choice of path affects other users.

If $\beta$ is negative, a user is spiteful: we know that adding a little more flow to $p$ will increase the social welfare cost of taking $p$ (the derivative $c'_e$ is positive), and since we negate this value, this lowers the user's perceived cost of taking $p$.
Conversely, if $\beta$ is positive, a user is altruistic: increasing flow increases the social welfare cost on $p$ and also the user's perceived cost of taking $p$.
We assume that $\beta$ ranges from -1 (extremely spiteful) to 1 (extremely altruistic), where $\beta=0$ corresponds to selfishness.

All analysis of the model assumes a particular distribution $\psi$ of $\beta$ for all users. We next present Chen and Kempe's core results when $\psi$ is uniform (all users have the same $\beta$ value) in arbitrary networks, and when $\psi$ is non-uniform in parallel-link networks.

\XXX{TODO: Prove that each instance has nash equilibrium?}

\subsubsection{Uniformly Distributed Altruism}
We first consider the case where $\psi$ is uniformly distributed, such that $\beta$ and therefore $c^\beta_p$ is the same for each user. We additionally assume that users tend to be altruistic, i.e., $\beta > 0$.
\begin{theorem}
For any input network $G$, demand rates $r$, and 
a uniform distribution $\psi$ with $\beta \in (0, 1]$,
if $c_e$ is nondecreasing and semi-convex for all $e$,
    then the price of anarchy is bounded by 
    $$\rho(G,r,c,\psi) \le \frac{1}{\beta}$$
\end{theorem}

\begin{proof-sketch}
This proof is short. Will add... just follows from definition.
\end{proof-sketch}

Chen and Kempe then address the problem of spite: how (uniformly) spiteful can users be before the PoA becomes infinite?
\begin{itemize}
\item\XXX{Maybe something about anarchy value and given class of cost functions? Could just mention (don't really need to give proof) since Tardos paper assumes same restrictions on cost function.}
\item\XXX{Not sure if we need this?}
\item\XXX{Could also just do the case for linear cost functions? Say it expands for any class}
\end{itemize}

\subsubsection{Arbitrarily Distributed Altruism}
We now consider the more realistic scenario: when users have an arbitrary distribution $\psi$ of altruism.
Chen and Kempe analyze bounds on the PoA in \emph{parallel link networks}, which have 
only one demand source-sink pair $(s,t)$ and (parallel) edges only between $s$ and $t$.
%Stackleberg routing shown to have unlimited PoA in single-commodity networks

\begin{theorem}
Convex + nondecreasing -> 1/(average altruism)
\end{theorem}

\begin{proof-sketch}
\XXX{TODO a bit complicated...}
\end{proof-sketch}

\begin{itemize}
\item \XXX{Result about centralized control?}
\item \XXX{Address spite (not possible to include negative support)}

%Almost all of the latency is incurred by a small fraction of spiteful users who together
%congest a link with very steep increase. At the same time, all altruistic users use links with very small constant latency.
%Then, the PoA is much larger than 1, while the bounds of both theorems would require it to be close to 1

\end{itemize}
