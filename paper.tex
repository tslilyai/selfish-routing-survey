\documentclass[acmlarge]{acmart}

\usepackage{booktabs} % For formal tables
\usepackage{xxxnotes} 

% Metadata Information
%\acmJournal{PACMHCI}
%\acmVolume{9}
%\acmNumber{4}
%\acmArticle{39}
%\acmYear{2010}
%\acmMonth{3}
%\acmArticleSeq{11}

%\acmBadgeR[http://ctuning.org/ae/ppopp2016.html]{ae-logo}
%\acmBadgeL[http://ctuning.org/ae/ppopp2016.html]{ae-logo}


% Copyright
%\setcopyright{acmcopyright}
%\setcopyright{acmlicensed}
%\setcopyright{rightsretained}
%\setcopyright{usgov}
%\setcopyright{usgovmixed}
%\setcopyright{cagov}
%\setcopyright{cagovmixed}

% DOI
%\acmDOI{0000001.0000001}

% Paper history
%\received{February 2007}
%\received{March 2009}
%\received[accepted]{June 2009}


% Document starts
\begin{document}
% Title portion
\title{The Price of Human Behaviors}

\author{Lillian Tsai}
\orcid{1234-5678-9012-3456}
\affiliation{%
  \institution{MIT}
  \city{Cambridge}
  \state{MA}
  \country{USA}}
\email{tslilyai@mit.edu}
\author{Vibhaalakshmi Sivaraman}
\affiliation{%
  \institution{MIT}
  \city{Cambridge}
  \state{MA}
  \country{USA}}
\email{vibhaa@mit.edu}
\author{Xiaoyue Gong}
\affiliation{%
  \institution{MIT}
  \city{Cambridge}
  \state{MA}
  \country{USA}}
\email{xygong@mit.edu}

\begin{abstract}
The impact of selfish actions on the latency incurred by a network users has historically been a problem of great interest to the algorithmic community. Our paper first presents a brief overview of the original selfish routing problem, the standard algorithms used to route in the selfish routing setting, and the limits on the optimality of these algorithms (termed as the ``price of anarchy"), and then compare and contrasts other recent works that have formulated the problem in
    (perhaps more realistic) settings, i.e., by taking into account the possibility for altruistic, risk averse, and diverse-interest behaviors.  This paper both presents and clarifies the findings of these papers in the context of the original selfish routing paper, and demonstrates how these papers' results can be synthesized into a more general framework addressing optimality of routing with various human behaviors or motivations.
\end{abstract}

\keywords{selfish routing, price of anarchy}

\maketitle

%%%%%%%%%%%%%%%%%%%%%%%%%%%%%%%%%%%%%%%%%%%%%%%%%%%%%%%%%%%%%%%%%%%%%%%%%%%%%%%%%%%%%%%%%%%%%%%%
%%%%%%%%%%%%%%%%%%%%%%%%%%%%%%%%%%%%%%%%%%%%%%%%%%%%%%%%%%%%%%%%%%%%%%%%%%%%%%%%%%%%%%%%%%%%%%%%
%%%%%%%%%%%%%%%%%%%%%%%%%%%%%%%%%%%%%%%%%%%%%%%%%%%%%%%%%%%%%%%%%%%%%%%%%%%%%%%%%%%%%%%%%%%%%%%%
\section{Introduction}
Finding the best strategy for network and traffic routing has historically been a problem of great importance combining the theoretical aspects of both game theory and computer science. 
\begin{itemize}
    \item Introduce noncooperative games / Nash equilibrium (in network setting)
    \item Briefly introduce selfish routing paper / price of anarchy
    \item Discuss how it is a limited view of human behavior
    \item Briefly discuss alternatives (papers including taxes, etc.)
    \item There are newer papers with more interesting versions of human behavior
    \item We present a survey of these newer papers to show how human behaviors affect price of anarchy
\end{itemize}

%%%%%%%%%%%%%%%%%%%%%%%%%%%%%%%%%%%%%%%%%%%%%%%%%%%%%%%%%%%%%%%%%%%%%%%%%%%%%%%%%%%%%%%%%%%%%%%%
%%%%%%%%%%%%%%%%%%%%%%%%%%%%%%%%%%%%%%%%%%%%%%%%%%%%%%%%%%%%%%%%%%%%%%%%%%%%%%%%%%%%%%%%%%%%%%%%
%%%%%%%%%%%%%%%%%%%%%%%%%%%%%%%%%%%%%%%%%%%%%%%%%%%%%%%%%%%%%%%%%%%%%%%%%%%%%%%%%%%%%%%%%%%%%%%%
\section{Background}
This section presents a brief history of traffic routing algorithms, defining the terminology
and the traffic routing problem.

\subsection{The Traffic Routing Problem}
Traffic routing problems naturally arise in communication or transportation networks, where
links in the network often becomes \emph{congested} if too many users decide to route their data 
or cars through that link. In these networks, the path each user chooses can affect the travel times of other
users. Here, we describe Roughgarden and Tardos' formalization of the problem as multicommodity flow networks~\cite{tardos,roughgarden}.

\medskip\noindent
\textbf{The input} to a traffic routing problem consists of:
\begin{itemize}
    \item A network $G = (V, E)$ of $|V|$ nodes (e.g., locations or servers) and $|E|$ links 
    \item A {rate} $r_i$ of traffic for each $(s_i,t_i)\in S$ representing the demanded amount of traffic from $s_i$ to $t_i$
    \item A {latency} cost function $c$ that assigns a per-edge function $c_e$ to each edge $e$ describing how adding traffic (i.e., congestion) to $e$ affects how long it takes to travel time across $e$. We can also think of $c$ as assigning per-path costs: for any path $p$ in the graph
        $$c_p(f) = \sum_{e\in P}c_e(f_e)$$ 
        We assume that $c$ is continuous, nonnegative, and nondecreasing.
    \item A set of $k$ source-destination pairs $S=\{(s_1,t_1), \cdots (s_k,t_k)\}$ representing traffic demands
\end{itemize}

\medskip\noindent
\textbf{Solutions} correspond to flow assignments to the set of simple paths $P_i$ between $s_i$ and $t_i$ for all $i$. To describe a flow assignment $f$, we can consider $f_p$, the flow on a single path $p \in P_i$ (this adds an equal amount of flow $f_p$ to all edges in $p$), as well as $f_e = \sum_p \sum_{e\in p} f_p$, the edge flow on edge $e$ (the sum of flow on all paths that use $e$).

   A \emph{feasible} solution given such an input is an assignment of path flows such that the demand from $s_i$ to $t_i$ is met:
$$\forall i,~\sum_{p\in P_i} f_p = r_i$$
   An \emph{optimal} (feasible) solution given such an input is the feasible flow assignment $f$ that minimizes the \textbf{total cost} $C(f)$, where
$$C(f) = \sum_i\sum_{p\in P_i}c_p(f)f_p = \sum_{e\in E} c_e(f_e)\cdot f_e$$
Intuitively, we are calculating the cost of each path of a given flow assignment, weighing 
each path's cost proportional to the amount of flow through the path.
There exists an optimal flow $f^*$ minimizing $C(f)$ because we assume $c$ is continuous and the set of feasible flows is compact.

\subsection{Coordination Models}
Before we can create algorithms to solve the traffic routing problem, we must first assume a \emph{coordination model} for our traffic network.
There are two clear extremes: (1) centralized control, in which some entity (e.g., an air traffic controller) knows all traffic demands and routes accordingly, and
(2) decentralization, i.e., a complete \emph{lack} of coordination between
entities in the network.

In the setting with centralized control, there is a clear optimal solution as described before.
However, in a decentralized and uncoordinated model, the lack of coordination can result in
inefficiencies. One such example of an uncoordinated model is the \emph{selfish routing} model, i.e. 
a model in which all entities in the network are selfish and choose a route minimizing their individual latency without caring (or knowing) about the effects on other users~\cite{tardos}.
In this selfish routing model, we can measure the ratio between the optimal and selfish outcomes, just as we had measured the distance from optimal of an approximation algorithm in a limited computational power model, and of online algorithms in an incomplete information model. 

\subsection{The Selfish Routing Model: The Price of Anarchy}
\XXX{TODO put results here}

\textbf{Analysis} 
TODO: talk about optimum and maybe price of anarchy? Need to move models first
%of a particular solution evaluates how inefficient the solution is compared to the optimum, i.e., by how much the latency exceeds the minimum possible latency.

%%%%%%%%%%%%%%%%%%%%%%%%%%%%%%%%%%%%%%%%%%%%%%%%%%%%%%%%%%%%%%%%%%%%%%%%%%%%%%%%%%%%%%%%%%%%%%%%
%%%%%%%%%%%%%%%%%%%%%%%%%%%%%%%%%%%%%%%%%%%%%%%%%%%%%%%%%%%%%%%%%%%%%%%%%%%%%%%%%%%%%%%%%%%%%%%%
%%%%%%%%%%%%%%%%%%%%%%%%%%%%%%%%%%%%%%%%%%%%%%%%%%%%%%%%%%%%%%%%%%%%%%%%%%%%%%%%%%%%%%%%%%%%%%%%
\section{Alternative Models}

\subsection{Altruistic}
\subsubsection{Model}
\subsubsection{Results}

%%%%%%%%%%%%%%%%%%%%%%%%%%%%%%%%%%%%%%%%%%%%%%%%%%%%%%%%%%%%%%%%%%%%%%%%%%%%%%%%%%%%%%%%%%%%%%%%
%%%%%%%%%%%%%%%%%%%%%%%%%%%%%%%%%%%%%%%%%%%%%%%%%%%%%%%%%%%%%%%%%%%%%%%%%%%%%%%%%%%%%%%%%%%%%%%%
%%%%%%%%%%%%%%%%%%%%%%%%%%%%%%%%%%%%%%%%%%%%%%%%%%%%%%%%%%%%%%%%%%%%%%%%%%%%%%%%%%%%%%%%%%%%%%%%
\subsection{Risk Adverse}
\subsubsection{Model}
\subsubsection{Results}

%%%%%%%%%%%%%%%%%%%%%%%%%%%%%%%%%%%%%%%%%%%%%%%%%%%%%%%%%%%%%%%%%%%%%%%%%%%%%%%%%%%%%%%%%%%%%%%%
%%%%%%%%%%%%%%%%%%%%%%%%%%%%%%%%%%%%%%%%%%%%%%%%%%%%%%%%%%%%%%%%%%%%%%%%%%%%%%%%%%%%%%%%%%%%%%%%
%%%%%%%%%%%%%%%%%%%%%%%%%%%%%%%%%%%%%%%%%%%%%%%%%%%%%%%%%%%%%%%%%%%%%%%%%%%%%%%%%%%%%%%%%%%%%%%%
\subsection{Diverse in Interests}
\subsubsection{Model}
\subsubsection{Results}


%%%%%%%%%%%%%%%%%%%%%%%%%%%%%%%%%%%%%%%%%%%%%%%%%%%%%%%%%%%%%%%%%%%%%%%%%%%%%%%%%%%%%%%%%%%%%%%%
%%%%%%%%%%%%%%%%%%%%%%%%%%%%%%%%%%%%%%%%%%%%%%%%%%%%%%%%%%%%%%%%%%%%%%%%%%%%%%%%%%%%%%%%%%%%%%%%
%%%%%%%%%%%%%%%%%%%%%%%%%%%%%%%%%%%%%%%%%%%%%%%%%%%%%%%%%%%%%%%%%%%%%%%%%%%%%%%%%%%%%%%%%%%%%%%%
\section{Discussion}
\subsection{Importance of Human Understanding}
\subsection{Future Work}
Clearly the models we present are a miniscule subset of all potential models for human behavior; 
furthermore, they are coarse-grained and oversimplistic in comparison with the complexity of the human
brain. As understanding of the neurophysiological aspects of human behavior improves, we hope to see a matching evolution in the precision and accuracy of these models for traffic routing as well.

\bibliography{paper}
\bibliographystyle{acm}

\end{document}
