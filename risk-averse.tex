\subsection{Risk Aversion}
The second model we consider accounts for the
tendency of users to pick routes with less variation in latency even if it comes at the cost of some added
latency on the paths chosen. This increase in latency can be quantified as the {\em{price of risk-aversion}}
which is the worst-case ratio of the latency or cost at a risk-averse Nash Equilibrium to that at a risk-neutral
Nash equilibrium or one where users are indifferent to variations in the latency itself.

%% The key result that we will be focussing on here is one where we show that the latency bounds in this context
%% is effectively the price of anarchy times the price of risk-aversion which intuitiveluy makes sense.

\subsubsection{Formalization} The {\textbf{formal model}} introduced in Lianeas et.al \cite{risk-averse} defines 
a risk aversion coefficient $\gamma$ that quantifies the users' tendency to prefer paths with less variability. 
A higher $\gamma$ means that one is more risk averse. The edge costs $c_e(x_e)$ now have a deterministic part 
or $l_e(x_e)$ and a noise modelled by a random variable $\xi_e(x_e)$. The latter is assumed to be independent across edges 
and has expectation $0$ and variance $\upsilon_e(x_e)$ for $x_e$ flow allowing us to sum them up over a path. To simplify 
the analysis, the model also defines $\upkappa$ to bound the variance-to-mean latency ratio. In other words, 
$\upsilon_e(x_e) \leq \upkappa l_e(x_e)$.
%Now,
%the expected latency and variance ($\upsilon_p$) on a path can be summed across all the edges. Consequently, our problem
%instance is $(G,r,l,\upsilon,\gamma)$ and we focus on $r = 1$ to simplify our analysis.

The cost function or the {\textbf{objective}} that every user is now optimizing for is of the form
$$c_p(f) = \sum_{e \in p}l_e(f_e) + \gamma \sum_{e \in p}\upsilon_e(f_e)$$
This function is assumed to be non-decreasing. Intuitively, the mean is identical to the original price of anarchy formulation
while the second term accounts for variance. Minimizing this implies that we want to minimize the variance depending on the
value of the risk coefficient itself.

If the maximum cost across some set of nash equlibria flows $x$ for the given problem instance 
that is restricted to some family of inputs and a fixed $\upkappa$ is $C(x)$, the {\textbf{price of risk-aversion}} is now defined as the
ratio $C(x)/C(z)$. Here $C(z)$ is the cost associated with some risk-neutral nash equilibrium $z$. 
In the following section, we look at bounding this price of risk-aversion. Note that this can be viewed as contributing a 
multiplicative factor to the price of anarchy in the overall change to the latency or cost of the system. We first prove a 
more basic result from an older paper on this topic \cite{risk-averse-background} and then proceed to the main result on the price
of risk-aversion for 
a special family of latency functions that are $(\lambda, \mu)$ smooth. Note that additional results can be proved for special classes
of graph topologies which can be found in the original paper \cite{risk-averse}.


%%\begin{itemize}
%%
%%\item Mean-variance objective 
%%\begin{itemize}
%%    \item additive factor allowing you to reason about optimal
%%    \item assumed to be non-decreasing for the same reasons as the original model
%%    \item the mean is similar to the old function we were minimizing or the latency alone 
%%    \item but now we have an additional term that we are trying to minimize the variance depending on our risk-aversion coefficient
%%    \item Players try to optimize for this mean-variance objective which takes both the above aspects into account which is collectively called path-cost
%%\end{itemize}
%%
%%\item Social Cost of a flow - sum of the expected latencies of all players $C(f) = \displaystyle \sum_{p \in P} f_pl_p(f) = 
%%\sum_{e \in E}f_el_e(f_e)$. This reoves the dependency on per user risk aversion coefficients and rather takes a look at the system as a whole.
%%
%%\item Define PRA 
%%\begin{itemize} 
%%    \item $\upkappa$ definition -  variance to mean ratio and maximum bound on it
%%    \item Maximum cost across some set of nash equlibria flows $x$ for the given problem instance that belongs 
%%	to a certain class of instances and $\upsilon(x) \leq \upkappa l(x)$
%%    \item note about results depending on the graph topology
%%\end{itemize}
%%\end{itemize}
%%


\subsubsection{Main Results}
\begin{theorem}
    If a flow $f$ is at a risk averse nash equlibrium and $f^*$ is any other flow, then $fC(f)\leq f^*C(f)$. 
    \label{variational}
\end{theorem}

\begin{proof}
    By definition, any flow at Nash Equilibrium routes on paths with minimum cost or only sends flow on a given path if its cost is less than
    the cost of sending the same flow on some other path. Thus, for a fixed demand and fixed path costs determined by $C(f)$, $f^*$ differs from $f$ in 
    atleast moving some $\epsilon$ flow from one lower cost path to another higher cost one which increases the total overall cost. 
    Consequently, let's say we have a flow $x$ that is a risk averse Nash Equlibrium 
    for the cost function $c_e = l_e(x_e) +\gamma \upsilon_e(x_e)$. If $z$ is a risk neutral Nash Equilibirium, it is still feasible for the risk averse
    mean-variance cost function, but is not the equilibrium in that scenario. By the above description and costs written as the sum across edges, we have
    $$\sum_{e \in E}x_e(l_e(x_e) + \gamma \upsilon(x_e)) \leq \sum_{e \in E} z_e(l_e(x_e) + \gamma \upsilon_e(x_e))$$
\end{proof}

\begin{definition}
    A latency function $l(x)$ is $(\lambda, \mu)$-smooth if for all $x, y \geq 0$ $$ yl(x) \leq \lambda yl(y) + \mu x l(x)$$
    
    This is a particular instance of smoothness definitions for functions that allows us to show bounds on the price of risk-aversion for such restricted families of functions. 
\end{definition}

\begin{theorem}
    The set of instances with latency functions ${l_e}_{e \in E}$ that are $(1,\mu)$-smooth around any risk-averse nash equilibrium $x_e$ for all $e \in E$ have price of risk-aversion
    $\leq \displaystyle \frac{(1 + \gamma \upkappa)}{(1 - \mu)}$
\end{theorem}

\begin{proof}
    This proof involves separating the edges into two sets $A$ and $B$ where $A$ contains edges whose flow $x_e$ in the risk-averse nash equilibrium is utmost the flow on the same edge
    $z_e$ in the risk-neutral nash equilibrium and $B$ contains the rest of the edges. $x$ continues to be a risk-averse nash equilibrium while $z$ is a risk-neutral one. 
    
    Let's consider the edges in $A$. We know that by definition, $\sum_{e \in A}l_e(x_e) \leq \sum_{e \in A}l_e(z_e)$. In turn this means 
    $$\sum_{e \in A}(1 + \gamma\upkappa)z_el_e(x_e) \leq \sum_{e \in A}(1 + \gamma \upkappa)z_el_e(z_e)$$.

    Let's similarly consider the edges in $B$. By the definition of $(1, \mu)$-smoothness, we have $$\sum_{e \in B}z_el_e(x_e) \leq \sum_{e \in B}z_el_e(z_e) + \mu x_el_e(x_e)$$.
    
    Together, the last two statements mean that (adding some terms two both of them to encompass all the edges in each type of term), 
    $$\sum_{e \in A}(1 + \gamma\upkappa)z_el_e(x_e) +  \sum_{e \in B}z_el_e(x_e) \leq \sum_{e \in E}z_el_e(z_e) + 
    \sum_{e \in E} \mu x_el_e(x_e) + \sum_{e \in E}(1 + \gamma \upkappa)z_el_e(z_e) = (1 + \gamma \upkappa)C(z) + \mu C(x) $$

    Now, if we are able to show that the total social cost $C(x)$ of the risk averse nash equilibrium flow is somehow utmost the expression above, we have established our proof
    by rearranging the terms, because the price of risk-aversion in this case is given by $C(x)/C(z)$. 

    To establish this, let's take the expression from the Proof of Theorem \ref{variational} and use $C(x) = \sum_{e \in E} x_el_e(x_e)$ as well as separate edges by sets $A$ and $B$, we have 
    $$C(x) + \sum_{e \in A} x_e\gamma\upsilon_e(x_e) + \sum_{e \in B} x_e\gamma\upsilon_e(x_e) \leq \sum_{e \in A} z_e\gamma\upsilon_e(x_e) + \sum_{e \in B} z_e\gamma\upsilon_e(x_e) + \sum_{e \in E} z_el_e(x_e)$$
    By the definitions of $A$ and $B$, we can extract the sum of second and third terms from the left hand side and the second term on the right hand size, because the former is larger than the latter
    and can't contribute to this inequality. If we separate the last term on the right into sets $A$ and $B$ and apply $\upsilon_e(x_e) \leq \upkappa l_e(x_e)$, we effectively are left with
    $$C(x) \leq \sum_{e \in A}(1 + \gamma\upkappa)z_el_e(x_e) +  \sum_{e \in B}z_el_e(x_e) \leq (1 + \gamma \upkappa)C(z) + \mu C(x) $$
    which proves exactly what we need. 
\end{proof}

\subsubsection{Similarities and Extensions} Note that this result is similar to that of the price of anarchy originally derived by Roughgarden and Tardos \cite{tardos}. If we assume no variation in prices or in other words, 
set $\upkappa = 0$ and consider linear latency functions which by definition are $(1, 1/4)$-smooth \cite{tardos-notes}, we get price-of-risk aversion is $\leq \displaystyle \frac{1}{1 - \mu} = 4/3$.

Further, this price of risk-aversion can be lower bounded for a specific case of a recursive Braess graph \XXX{have we mentioned this before} and the gap between the upper and
lower bounds can be more neatly quantified. It can also be exactly computed for a series-parallel recursive graph to be $1 + \gamma \upkappa$. The details of these proofs can be found
in \cite{risk-averse}.
