\subsection{Risk Aversion}
The second model we consider is one investigated by Lianeas et.al \cite{risk-averse} which accounts for the
tendency of users to pick routes with less variation in latency even if it comes at the cost of some added
latency on the path as a whole. This increase in latency can be quantified as the {\em{price of risk-aversion}}
which is the worst-case ratio of the latency or cost at a risk-averse Nash equlibrium to that at a risk-neutral
Nash equilibrium. 

%% The key result that we will be focussing on here is one where we show that the latency bounds in this context
%% is effectively the price of anarchy times the price of risk-aversion which intuitiveluy makes sense.

\subsubsection{Formalization}
- define risk aversion coefficient - the higher the more risk averse because you're minimizing objective
- Mean-variance objective - additive factor allowing you to reason about optimal
- K definition - mean to variance objective
- note about results depending on the graph topology


\subsubsection{Main Results}
