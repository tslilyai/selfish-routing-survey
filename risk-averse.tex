\subsection{Risk-aversion}
The second model we consider accounts for the tendency of users to pick routes with less variation in latency even if it comes at the cost of some added latency. This increase in latency can be quantified as the {\em{price of risk-aversion}},
the worst-case ratio of the latency at a risk-averse Wardrop equilibrium to that at a risk-neutral
Wardrop equilibrium or one where users are indifferent to variations in the latency itself. The latter is exactly the original problem forumated in Section. 2.3.

%% The key result that we will be focusing on here is one where we show that the latency bounds in this context
%% is effectively the price of anarchy times the price of risk-aversion which intuitiveluy makes sense.

\subsubsection{Formalization} The formal model introduced in Lianeas et.al~\cite{risk-averse} defines 
a risk-aversion coefficient $\gamma$ that quantifies the users' tendency to prefer paths with less variability. 
A higher $\gamma$ means that one is more risk-averse. Lianeas et.al define a new individual user cost function, where 
the cost on each edge includes both the deterministic $\ell_e(x_e)$ and a noise modeled by a random variable $\xi_e(x_e)$. The noise has expectation $0$ and variance $\upsilon_e(x_e)$ for $x_e$ flow. Furthermore, we assume that it is independent across edges, allowing us to sum the variance and mean of the noise over all edges in a path. To simplify the analysis, the model also defines a bound $\upkappa$ on the variance-to-mean latency ratio:
$\upsilon_e(x_e) \leq \upkappa \ell_e(x_e)$.
%Now,
%the expected latency and variance ($\upsilon_P$) on a path can be summed across all the edges. Consequently, our problem
%instance is $(G,r,l,\upsilon,\gamma)$ and we focus on $r = 1$ to simplify our analysis.
Thus, the individual cost function for each user on a given path $P$ is of the form
$$c^\gamma_P(f) = \sum_{e \in P}\ell_e(f_e) + \gamma \sum_{e \in P}\upsilon_e(f_e)$$
%and the resulting Wardrop equilibrium is achieved at the flow solutions to the program
%$$NE^\gamma= \min_f\sum_e\int_0^{f_e}c_e^\gamma(t)dt \text{ subject to feasibility constraints}$$

$c^\gamma_P$ is assumed to be non-decreasing with increasing inputs. Intuitively, the mean value of $c^\gamma_P(f)$ is identical to the cost given the selfish model's individual user cost formulation, while the second term accounts for variance. Minimizing this implies that we want to minimize the variance at the expense of a larger mean depending on the value of the risk coefficient itself.

If we let $C(x)$ be the the maximum cost across some set $x$ of flows at the Wardrop equilibrium for a given problem instance 
(restricted to some family of inputs and a fixed $\upkappa$), the {\textbf{price of risk-aversion}} is now defined as the
ratio $C(x)/C(z)$. Here $C(z)$ is the cost associated with some risk-neutral Wardrop equilibrium $z$. 
In the following section, we look at bounding this price of risk-aversion. 
Bounds on the price of risk-aversion are equivalent to bounding 
a multiplicative factor of the price of anarchy: the price of risk-aversion tells us how many times worse 
the price of anarchy can become.
%Note that this can be viewed as contributing a 
%multiplicative factor to the price of anarchy in the overall change to the latency or cost of the system. 
We first prove a 
more basic result from an older paper on this topic~\cite{risk-averse-background} and then proceed to the main result on the price
of risk-aversion for 
a special family of latency functions that are $(\lambda, \mu)$ smooth. Lianeas et.al also prove additional results for special classes
of graph topologies, which we leave for future reading~\cite{risk-averse}.


%%\begin{itemize}
%%
%%\item Mean-variance objective 
%%\begin{itemize}
%%    \item additive factor allowing you to reason about optimal
%%    \item assumed to be non-decreasing for the same reasons as the original model
%%    \item the mean is similar to the old function we were minimizing or the latency alone 
%%    \item but now we have an additional term that we are trying to minimize the variance depending on our risk-aversion coefficient
%%    \item Players try to optimize for this mean-variance objective which takes both the above aspects into account which is collectively called path-cost
%%\end{itemize}
%%
%%\item Social Cost of a flow - sum of the expected latencies of all players $C(f) = \displaystyle \sum_{p \in P} f_P\ell_P(f) = 
%%\sum_{e \in E}f_e\ell_e(f_e)$. This reoves the dependency on per user risk-aversion coefficients and rather takes a look at the system as a whole.
%%
%%\item Define PRA 
%%\begin{itemize} 
%%    \item $\upkappa$ definition -  variance to mean ratio and maximum bound on it
%%    \item Maximum cost across some set of Wardrop equlibria flows $x$ for the given problem instance that belongs 
%%	to a certain class of instances and $\upsilon(x) \leq \upkappa \ell(x)$
%%    \item note about results depending on the graph topology
%%\end{itemize}
%%\end{itemize}
%%

\paragraph{Equilibrium flows} All equilibrium flows that we discuss for the following models are with respect to the {\it Wardrop equilibrium}. A flow $f$ is at the Wardrop equilibrium of an instance if for every $k\in K$, for every path $P\in \mathcal{P}_k$ with postive flow, and any diversity parameter $\omega$ on it, the path cost $c_P^\omega(f)\le c_{P'}^r(f)$ for all paths $P'\in \mathcal{P}_k$. It is known that a Nash equilibrium in a network game with a finite number of players converges to a Wardrop equilibrium in a network game with a finite number of players converges to a Wardrop equilibrium when the number of players increases \cite{haurie}.


\subsubsection{Main Results}
\begin{theorem}
    If a flow $\hat{f}$ is at a risk-averse Wardrop equilibrium and $f^*$ is any other flow, then $\hat{f}C(\hat{f})\leq f^*C(\hat{f})$. 
    \label{variational}
\end{theorem}

\begin{proof}
    By definition, any flow at Wardrop equilibrium routes on paths with minimum cost or only sends flow on a given path if its cost is less than
    the cost of sending the same flow on some other path (with respect to the individual user costs). 
    Thus, any other flow $f^*$ that routes some $\epsilon > 0$ flow differently from $\hat{f}$ will increase the total cost of the flow over all individual users' costs.

    Consequently, let flow $x$ be at risk-averse Wardrop equilibrium for the individual cost function $c^\gamma_e = \ell_e(x_e) +\gamma \upsilon_e(x_e)$. Let $z$ be at risk-neutral Wardrop equilibirium (a feasible flow for the risk-averse mean-variance cost function). 
    Clearly, $z$ is not at Wardrop equilibrium for $c^\gamma$ when $\gamma \neq 0$. 
    By the above description and costs written as the sum across edges, we have
    $$\sum_{e \in E}x_e(\ell_e(x_e) + \gamma \upsilon(x_e)) \leq \sum_{e \in E} z_e(\ell_e(x_e) + \gamma \upsilon_e(x_e))$$
\end{proof}

\begin{definition}
    A latency function $\ell(x)$ is $(\lambda, \mu)$-smooth if for all $x, y \geq 0$ $$ y\ell(x) \leq \lambda y\ell(y) + \mu x \ell(x)$$
    This particular smoothness definition defines a class of functions for which we can bound the price of risk-aversion.
\end{definition}

\begin{theorem}
    The set of instances with latency functions ${\ell_e}_{e \in E}$ that are $(1,\mu)$-smooth around any risk-averse Wardrop equilibrium $x_e$ for all $e \in E$ have price of risk-aversion
    $\leq \displaystyle \frac{(1 + \gamma \upkappa)}{(1 - \mu)}$
\end{theorem}

\begin{proof}
    This proof involves separating the edges into two sets $A$ and $B$. $A$ contains edges whose flow $x_e$ at risk-averse Wardrop equilibrium is at most the flow $z_e$ on the same edge $e$ at risk-neutral Wardrop equilibrium; $B$ contains the rest of the edges. 
    As before, we let flow $x$ to be at risk-averse Wardrop equilibrium, and flow $z$ to be at risk-neutral Wardrop equilibrium. 
    
    Let's consider the edges in $A$. We know that by definition, $\sum_{e \in A}\ell_e(x_e) \leq \sum_{e \in A}\ell_e(z_e)$, implying that 
    $$\sum_{e \in A}(1 + \gamma\upkappa)z_e\ell_e(x_e) \leq \sum_{e \in A}(1 + \gamma \upkappa)z_e\ell_e(z_e)$$

    Let's similarly consider the edges in $B$. By the definition of $(1, \mu)$-smoothness, we have $$\sum_{e \in B}z_e\ell_e(x_e) \leq \sum_{e \in B}z_e\ell_e(z_e) + \mu x_e\ell_e(x_e)$$
    
    Taken together, with the addition of some terms to encompass all the edges in the graph under each type of term, we can conclude that
    $$\sum_{e \in A}(1 + \gamma\upkappa)z_e\ell_e(x_e) +  \sum_{e \in B}z_e\ell_e(x_e) \leq \sum_{e \in E}z_e\ell_e(z_e) + 
    \sum_{e \in E} \mu x_e\ell_e(x_e) + \sum_{e \in E}(1 + \gamma \upkappa)z_e\ell_e(z_e) = (1 + \gamma \upkappa)C(z) + \mu C(x) $$

    Now, if we are able to show that the total social welfare cost $C(x)$ of the risk-averse Wardrop equilibrium flow is at most the above expression, we have proven our desired result: we can rearrange the terms to get the price of risk-aversion $C(x)/C(z)$. 
    
    If we take the expression from the proof of Theorem \ref{variational}, use $C(x) = \sum_{e \in E} x_e\ell_e(x_e)$, and consider edges in sets $A$ and $B$ separately, we get
    $$C(x) + \sum_{e \in A} x_e\gamma\upsilon_e(x_e) + \sum_{e \in B} x_e\gamma\upsilon_e(x_e) \leq \sum_{e \in A} z_e\gamma\upsilon_e(x_e) + \sum_{e \in B} z_e\gamma\upsilon_e(x_e) + \sum_{e \in E} z_e\ell_e(x_e)$$
    By the definitions of $A$ and $B$, we can drop the sum of second and third terms on the LHS and the second term on the RHS because the former is larger than the latter and does not contribute to this inequality. If we separate the last term on the RHS into sets $A$ and $B$ and apply $\upsilon_e(x_e) \leq \upkappa \ell_e(x_e)$, we effectively are left with
    $$C(x) \leq \sum_{e \in A}(1 + \gamma\upkappa)z_e\ell_e(x_e) +  \sum_{e \in B}z_e\ell_e(x_e) \leq (1 + \gamma \upkappa)C(z) + \mu C(x) $$
    which proves exactly what we need. The last inequality is taken from the expression above.
\end{proof}

\subsubsection{Similarities and Extensions} Note that the result of Theorem 10 is similar to that of the price of anarchy in the selfish model derived by Roughgarden and Tardos~\cite{tardos} and demonstrated in Section~\ref{sec:background}, Theorem~\ref{thm:linear}. If we assume no variation in prices or in other words, 
set $\upkappa = 0$ and consider linear latency functions which by definition are $(1, 1/4)$-smooth~\cite{tardos-notes}, we get that the price of risk-aversion is $\leq \displaystyle \frac{1}{1 - \mu} = 4/3$.

Furthermore, this price of risk-aversion can be lower bounded for a specific case of a recursive Braess graph (Figure~\ref{fig:braess}) and the gap between the upper and
lower bounds can be more neatly quantified. It can also be exactly computed for a series-parallel recursive graph to be $1 + \gamma \upkappa$. The details of these proofs can be found
in~\cite{risk-averse}.
