\subsection{Risk Aversion}
The second model we consider is one investigated by Lianeas et.al \cite{risk-averse} which accounts for the
tendency of users to pick routes with less variation in latency even if it comes at the cost of some added
latency on the path as a whole. This increase in latency can be quantified as the {\em{price of risk-aversion}}
which is the worst-case ratio of the latency or cost at a risk-averse Nash equlibrium to that at a risk-neutral
Nash equilibrium. 

%% The key result that we will be focussing on here is one where we show that the latency bounds in this context
%% is effectively the price of anarchy times the price of risk-aversion which intuitiveluy makes sense.

\subsubsection{Formalization}
\begin{itemize}
\item Problem instance
    \begin{itemize}
        \item define risk aversion coefficient $\gamma$ - the higher the more risk averse because you're minimizing objective
        \item edge latencies no longer modelled as a determininistic function but rather with a deterministic part $l_e(f_e)$ and
            a random variable $\xi_e(f_e)$ that represents the noise on the delay. This part is assumed to be independent across edges and has expectation
            $0$ and variance $\sigma^2_e(f_e)$.
        \item Now the expected latency on a path and the variance of the path ($\upsilon_p$) is just the sum of the expected delays and variances across all edges
            on the path.
        \item Problem is now $(G,r,l,\upsilon,\gamma)$ and we focus on $r = 1$ for a simpler scenario

    \end{itemize}
\item Mean-variance objective 
    \begin{itemize}
        \item additive factor allowing you to reason about optimal
        \item assumed to be non-decreasing for the same reasons as the original model
        \item the mean is similar to the old function we were minimizing or the latency alone 
        \item but now we have an additional term that we are trying to minimize the variance depending on our risk-aversion coefficient
        \item Players try to optimize for this mean-variance objective which takes both the above aspects into account which is collectively called path-cost
    \end{itemize}

\item Social Cost of a flow - sum of the expected latencies of all players $C(f) = \displaystyle \sum_{p \in P} f_pl_p(f) = 
    \sum_{e \in E}f_el_e(f_e)$. This reoves the dependency on per user risk aversion coefficients and rather takes a look at the system as a whole.

\item Define PRA 
    \begin{itemize} 
        \item $\upkappa$ definition -  variance to mean ratio and maximum bound on it
        \item Maximum cost across some set of nash equlibria flows $x$ for the given problem instance that belongs 
            to a certain class of instances and $\upsilon(x) \leq \upkappa l(x)$
        \item note about results depending on the graph topology
    \end{itemize}
\end{itemize}


\subsubsection{Main Results}
