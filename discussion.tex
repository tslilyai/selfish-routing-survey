\section{Discussion}\label{sec:discussion}

\begin{table}[h]
\begin{center}
    \begin{tabular}{|p{2cm}| p{7cm} | p{8cm}|} 
 \hline
        \begin{center}Name\end{center} & \begin{center}Objective\end{center} & \begin{center} Results\end{center} \\
 \hline\hline
        Social Welfare & $SW = \min_f\Big(\sum_e f_e\ell_e(f_e)\Big)$ & $\rho(G,r,\ell) = 1$ (optimal by definition) \\
 \hline
     Selfish & $NE^s = \min_f\Big(\sum_e\int_0^{f_e} \ell_e(t)dt\Big)$ & 
        $\bullet~\rho(G,r,\ell) \le \frac{C(f^*_2)}{C(f^*)}$, where $f^*_2$ is an flow optimizing $SW$ for input $\rho(G,2r,\ell)$, and $f^*$ is an flow optimizing $SW$ for input $\rho(G,r,\ell)$.\\ 
        &  & $\bullet~\rho(G,r,\ell) \le 4/3$ when $\ell$ is linear\\
 \hline
        Altruistic & $NE^\beta = \min_f\Big(\sum_e\int_0^{f_e} \ell_e(t) + \beta tl'e(t)dt\Big)$ & 
        $\bullet~\rho(G,r,\ell,\psi) \le \frac{1}{\beta}$ when $\psi$ is a uniform distribution of $\beta >0$\\
        & & $\bullet~\rho(G,r,\ell,\psi) = \infty$ when $\psi$ is a uniform distribution of $\beta < \frac{-1}{d}$ and $\ell$ is a polynomial of degree $\le d$\\
        & & $\bullet~\rho(G,r,\ell,\psi) \le \frac{1}{\bar{\beta}}$ when $G$ is parallel-link and $\psi$ is any distribution of $\beta \ge 0$ with average altruism $\bar{\beta}$\\
\hline
     Risk-averse & $NE^\gamma = \min_f\Big(\sum_e\int_0^{f_e} \ell_e(t) + \gamma\upsilon(t)dt\Big)$ & TODO\\
\hline
     Diverse Interests & $NE^\omega = \min_f\Big(\sum_e\int_0^{f_e} \ell_e(t) + \omega\cdot \sigma_e dt\Big)$ & TODO\\
     & where $\ell$ and $\sigma$ are the costs of two general criterion &\\
\hline
\end{tabular}
\end{center}
    \caption{Comparing the optimization functions and price of anarchy results for the behavior models discussed in the paper}
    \label{tab:comparison}
\end{table}

\subsection{Extensions}

The models we covered are quite limited and coarse-grained compared to the complexity of human behavior. The most general of the models, the diversity model, only explored the realm of having at most two criteria in total in the objective functions of the whole diverse network.

\subsection{Applications}
Although the knowledge we have of the affect of human nature on the optimality of a network is still quite limited, we can nevertheless leverage this knowledge to improve our decision making. Here, we discuss three major ways to do so.

\subsubsection{Policy Design}
As we discussed in the diverse model section, diverse selfish routing models help us understand how we can leverage policies and the natural diversity of goals in a network to increase the social welfare and efficiency of the network as a whole. 
We mentioned the example of tolls helping to implement the social optimum at equilibrium when all agents have the same goal, where the goal is a linear combination of time and money~\cite{beckmann1956studies}. We can add many more factors that agents will consider, as well as analyse more complex relationships between these factors, if we can understand the effects of these factors on users' decisions in the network.

\subsubsection{Information Distribution}
To enable diversity among acting agents, we must be able to provide a variety of information to the agents. For example, to utilize the time-money tradeoff, the agents must be informed of the estimated time of taking each route, and estimated expense of taking each route. 

The implications of information distribution go much further beyond the time-money tradeoff, however. Imagine if we have a route A that is fast but heavy traffic on that route does a lot of damage to the environment, and another route B that is slower but has less impact on the environment. To make the environmental-concern even a criterion for any subset of the heterogeneous agents, they need to first be aware that route A does more damage to the environment. Being exposed to new information about the network is one of the most effective ways to allow the agents to possibly include multiple criteria in their objective functions. Thus, there is intrinsic value in providing more information on the network to the agents.

\subsubsection{Network Design}

As we have found that, under different assumptions of the agent behavior, we can design the network we have into different structures, in order to make sure that the affect of the previous two methods are positive. For example, as we have shown in the diversity model, we can design our network to make sure that no Braess graph can be embedded in our network, and our network is block-matching.
These important applications also point at the directions for potential future research. 

%Clearly the models we present are a miniscule subset of all potential models for human behavior; 
%furthermore, they are coarse-grained and oversimplistic in comparison with the complexity of the human
%brain. As understanding of the neurophysiological aspects of human behavior improves, we hope to see a matching evolution in the precision and accuracy of these models for traffic routing as well.
